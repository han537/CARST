\documentclass[12pt]{article}
\usepackage{graphicx}
\usepackage{longtable}
\usepackage{lineno}
\usepackage{natbib}
\usepackage[margin=1in]{geometry}

\newenvironment{myindentpar}[1]%
{\begin{list}{}%
 {\setlength{\leftmargin}{#1}}%
  \item[]%
 }
{\end{list}}

\title{Landsat Pixel-Tracking Using ``landsatPX.py''}


%\bibliographystyle{agufull08}

\author{Andrew Kenneth Melkonian}
\begin{document}

\maketitle

\vspace{1pt}
\hrule
\vspace{2pt}


\noindent {\bf \underline{Description}}:

Performs pixel-tracking on Landsat imagery.
Assumes imager is already orthorectified (e.g. L1G/T product for Landsat 7, 8).
Motion-elevation correction requires a DEM.

\noindent {\bf \underline{Dependencies}}:

\noindent {\bf Packages/Software:} \\
\indent Python (including numpy, scipy, pylab, matplotlib) \\
\indent ROI\_PAC (``ampcor'') \\
\indent GDAL \\
\indent GMT \\
\indent sed \\
\indent gawk \\
\indent IDL/ENVI (if pre-filtering using ``gausshighpassfiltgausstretch.pro'') \\
\indent MATLAB (if using coherency filter for post-filtering)

\noindent {\bf Scripts:} \\
\indent splitAmpcor.py \\
\indent findOffset.py \\
\indent getxyzs.py \\
\indent motionElevCorrection.py (if performing motion-elevation correction, must be in the SAME directory as highResPX.py) \\
\indent gausshighpassfiltgausstretch.pro (if pre-filtering using this script you must also have IDL/ENVI) \\
\indent noiseremoval.m (if using coherency filter for post-filtering) \\
\indent remloners.m (if using coherency filter for post-filtering) \\
\indent remnoise.m (if using coherency filter for post-filtering) \\
\indent grdread2.m (if using coherency filter for post-filtering) \\
\indent grdwrite2.m (if using coherency filter for post-filtering) \\
\indent NOTE: all MATLAB scripts must be in the ``MATLAB'' directory specified in the parameters file

\noindent {\bf \underline{Usage:}}

python landsatPX.py /path/to/landsat\_params.txt

\noindent {\bf \underline{Parameters:}}

/path/to/landsat\_params.txt: Path to ASCII text file defining all the necessary parameters.

\noindent {\bf Example parameters file:} \\
\noindent ::::::::::::::::::::::::::: \\
\noindent {\bf /path/to/landsat\_params.txt} \\
\noindent ::::::::::::::::::::::::::: \\
UTM\_ZONE       = 7 \\
UTM\_LETTER     = V \\
ICE             = /home/user/Region\_utm7v\_ice.gmt \\
ROCK            = /home/user/Region\_utm7v\_rock.gmt \\
METADATA\_DIR   = /home/user/Landsat8/Images \\
PAIRS\_DIR      = /home/user/Landsat8/Pairs \\
PROCESSORS      = 4 \\
RESOLUTION      = 15 \\
SNR\_CUTOFF     = 0 \\
DEM\_GRD        = /home/user/DEMs/DEM\_utm7v.grd \\
PREFILTER       = False \\
REF\_X          = 32 \\
REF\_Y          = 32 \\
SEARCH\_X       = 32 \\
SEARCH\_Y       = 32 \\
STEP            = 8 \\
M\_SCRIPTS\_DIR = /home/user/MATLAB/scripts \\
VEL\_MAX        = 10 \\
TOL             = 0.2 \\
NUMDIF          = 3 \\
SCALE           = 1500000 \\
PAIRS           = /home/user/Landsat8/Pairs/landsat8\_pairs.txt \\

\noindent {\bf Parameter descriptions:}

\noindent {\bf UTM\_ZONE:} Number of UTM zone to use (imagery will be reprojected to this zone if not already projected in it)

\noindent {\bf UTM\_LETTER:} Letter of UTM zone (for use by GMT scripts)

\noindent {\bf ICE:} Path to GMT polygon file containing glacier outlines projected in UTM\_ZONE.
Outlines for many regions available via Global Land Ice Measurements from Space (GLIMS, www.glims.org).

\noindent {\bf ROCK:} Path to GMT polygon file containing internal rock outlines projected in UTM\_ZONE.

\noindent {\bf METADATA\_DIR:} Directory containing metadata, metadata files MUST have the same identifier as image files
e.g. if image file is LC81760062014211LGN00\_B8.TIF, metadata file MUST be LC81760062014211LGN00\_MTL.txt

\noindent {\bf PAIRS\_DIR:} Directory where all file created by this script will be written.
A separate directory will be created for each pair, with the following format: YYYYMMDDHHMMSS\_YYYYMMDDHHMMSS
The first date/time-stamp corresponds to the later image in the pair, the second to the earlier image
DO NOT MODIFY the pair directory names created by the script if you expect the script to work

\noindent {\bf PROCESSORS:} Number of processors to use when running ``ampcor"
To find the number of ``processors'' available on your machine (if you are using linux), enter the command ``top'' then type the number ``1"
NOTE: You can enter any number here, but if that number is higher than the number of ``processors'' available on your machine you will not see a performance increase.

\noindent {\bf RESOLUTION:} Resolution of the input imagery in meters (e.g. ``15'' for band 8 of Landsat 7 and 8).

\noindent {\bf SNR\_CUTOFF:} Signal-to-noise ratio threshold, *\_filt.grd velocity grids created by the script during post-processing will be clipped so that all pixels
with an SNR below this threshold are masked out.

\noindent {\bf DEM\_GRD:} Path to NetCDF grid file of elevations projected in UTM\_ZONE.
MAKE SURE this covers the maximum extent of the imagery if you plan on using motion-elevation corrected results.
Motion-elevation correction cannot be performed where the elevation is unknown.

\noindent {\bf PREFILTER:} Set to True if you are using IDLDE to perform gaussian high-pass filtering using the gausshighpassgausstretch.pro script.
If you set this to ``True'' DO NOT modify the file names created by the *.pro script, let landsatPX.py deal with it.
NOTE: This is often not necessary for Landsat 7 and 8.

\noindent {\bf REF\_X:} Reference window size in pixels for the longitude/width/columns/samples direction.
NOTE: 32 has proved successful, but you may wish to modify, e.g. a smaller window for smaller glaciers/faster processing.

\noindent {\bf REF\_Y:} Reference window size in pixels for the latitude/length/rows/lines direction.
NOTE: 32 has proved successful, but you may wish to modify, e.g. a smaller window for smaller glaciers/faster processing.

\noindent {\bf SEARCH\_X:} Search window size in pixels for the longitude/width/columns/samples direction.
Set this so that it is twice the maximum offset (maximum velocity multiplied by time interval, then divided by RESOLUTION).

\noindent {\bf SEARCH\_Y:} Search window size in pixels for the latitude/length/rows/lines direction.
Set this so that it is twice the maximum offset (maximum velocity multiplied by time interval, then divided by RESOLUTION).

\noindent {\bf STEP:} Pixel-tracking is performed every STEP pixels.
The smaller this number, the higher resolution the velocity results will be and the longer ``ampcor'' will take to run.

\noindent {\bf M\_SCRIPTS\_DIR:} Directory containing the Matlab scripts ``noiseremoval.m", ``remloners.m", ``remnoise.m", ``grdread2.m", ``grdwrite2.m"
NOTE: This is not necessary for processing.

\noindent {\bf VEL\_MAX:} Maximum velocity threshold (in m/day) for use by ``noiseremoval.m".
Velocities above this value will be clipped for *\_filt.grd files.
Lower values filter out MORE, higher values filter out LESS.
NOTE: This applies to E-W and N-S velocities, NOT to overall speed.

\noindent {\bf TOL:} Adjacent pixels whose velocity is within TOL*VEL\_MAX are considered  ``similar'' pixels.
For example, if TOL is 0.2 and VEL\_MAX is 10, pixels that are with 2 m/day of each other are considered ``similar".
Lower values filter out MORE, higher values filter out LESS.
This is used in conjunction with NUMDIF (see below).

\noindent {\bf NUMDIF:} The minimum number of ``similar'' pixels that must ``adjacent'' to the pixel being considered.
If less than this number of ``similar'' adjacent pixels are found, the pixel will be filtered out of the *\_filt.grd velocity results.
Lower values filter out LESS, higher values filter out MORE.

\noindent {\bf SCALE:} Scale to be used by GMT when plotting results, 1500000 often works for an entire Landsat 8 image.

\noindent {\bf PAIRS:} Path to two-column ASCII text file listing image pair paths.
Each row corresponds to an image pair, containing two entries that are paths to each image in the pair.

\noindent {\bf \underline{Output:}}

A directory will be created in the ``PAIRS\_DIR'' directory for each valid pair given in the ``PAIRS'' file with the following format: \\
\indent {\bf YYYYMMDDHHMMSS\_YYYYMMDDHHMMSS} \\
The first date/time-stamp corresponds to the later image in the pair, the second to the earlier image.
DO NOT MODIFY the pair directory names created by the script or the filenames created by this script if you expect the script to work.

\noindent The following results will be created in these directories if the script runs sucessfully: \\
\indent {\bf YYYYMMDDHHMMSS\_YYYYMMDDHHMMSS\_*\_eastxyz.txt:} UTM Easting, UTM Northing, E-W offset in meters, SNR (space-delimited) \\
\indent {\bf YYYYMMDDHHMMSS\_YYYYMMDDHHMMSS\_*\_northxyz.txt:} UTM Easting, UTM Northing, N-S offset in meters, SNR (space-delimited) \\
\indent These are 4-column ASCII text files containing the E-W and N-S OFFSETS in METERS (NOT velocities).
The first column is longitude (UTM easting), the second is latitude (UTM northing), the third is offset in meters, the fourth is signal-to-noise ratio.

\noindent Unfiltered NetCDF grid files containing VELOCITIES in METERS PER DAY will be created from the ASCII text files: \\
\indent {\bf YYYYMMDDHHMMSS\_YYYYMMDDHHMMSS\_*\_eastxyz.grd:} E-W velocities (NOTE: MAY NEED TO MULTIPLY BY -1) \\
\indent {\bf YYYYMMDDHHMMSS\_YYYYMMDDHHMMSS\_*\_northxyz.grd:} N-S velocities \\
\indent {\bf YYYYMMDDHHMMSS\_YYYYMMDDHHMMSS\_*\_mag.grd:} Speeds from E-W and N-S velocities \\
\noindent A NetCDF grid file containing the SNRs will be created as well:
\indent {\bf YYYYMMDDHHMMSS\_YYYYMMDDHHMMSS\_*\_snrxyz.grd:} SNRs \\
\noindent An image of the speeds will be created, as well as a file containing the GMT commands used to create the image (if you'd like to modify it to create your own image): \\
\indent {\bf YYYYMMDDHHMMSS\_YYYYMMDDHHMMSS\_mag.pdf:} PDF image of speeds \\
\indent {\bf YYYYMMDDHHMMSS\_YYYYMMDDHHMMSS\_image.gmt:} GMT commands used to create ``YYYYMMDDHHMMSS\_YYYYMMDDHHMMSS\_mag.pdf''

\noindent If filtering scripts are present, filtered files will be created corresponding to each of the unfiltered files: \\
\indent {\bf YYYYMMDDHHMMSS\_YYYYMMDDHHMMSS\_*\_eastxyz\_filt.grd:} Filtered E-W velocities \\
\indent {\bf YYYYMMDDHHMMSS\_YYYYMMDDHHMMSS\_*\_northxyz\_filt.grd:} Filtered N-S velocities \\
\indent {\bf YYYYMMDDHHMMSS\_YYYYMMDDHHMMSS\_*\_mag\_filt.grd:} Speeds from filtered E-W and N-S velocities \\
\indent {\bf YYYYMMDDHHMMSS\_YYYYMMDDHHMMSS\_*\_mag\_filt.pdf:} PDF image of filtered speeds \\
\indent {\bf YYYYMMDDHHMMSS\_YYYYMMDDHHMMSS\_*\_image\_filt.gmt:} GMT commands used to create ``YYYYMMDDHHMMSS\_YYYYMMDDHHMMSS\_mag\_filt.pdf''

\noindent If a DEM is specified, motion-elevation corrected files will be created from either the filtered (if present) or unfiltered velocities: \\
\indent {\bf YYYYMMDDHHMMSS\_YYYYMMDDHHMMSS\_*\_eastxyz*\_corrected.grd:} Filtered E-W velocities \\
\indent {\bf YYYYMMDDHHMMSS\_YYYYMMDDHHMMSS\_*\_northxyz*\_corrected.grd:} Filtered N-S velocities \\
\indent {\bf YYYYMMDDHHMMSS\_YYYYMMDDHHMMSS\_*\_mag*\_corrected.grd:} Speeds from filtered E-W and N-S velocities \\
\indent {\bf YYYYMMDDHHMMSS\_YYYYMMDDHHMMSS\_*\_mag*\_corrected.pdf:} PDF image of filtered speeds \\
\indent {\bf YYYYMMDDHHMMSS\_YYYYMMDDHHMMSS\_*\_image*\_corrected.gmt:} GMT commands used to create ``YYYYMMDDHHMMSS\_YYYYMMDDHHMMSS\_mag\_filt.pdf'' \\

\end{document}
